%author : Raimana Bouissou


\documentclass[a4paper,12pt]{article}
\usepackage[frenchb]{babel}
\usepackage[utf8]{inputenc}
\usepackage[T1]{fontenc}
\usepackage{amssymb}
\usepackage{amsmath}
\usepackage[pdftex]{graphicx}
\usepackage{titlesec}


\usepackage[pdftex,bookmarks,colorlinks]{hyperref}
\hypersetup{colorlinks,%
citecolor=black,%
filecolor=black,%
linkcolor=black,%
urlcolor=black,%
pdftex
}

% Raccourcis...
\newcommand{\psubject}{Algorithme de Naimi-Tréhel et tolérance aux pannes}
\newcommand{\nt}{Naimi-Tréhel }
\newcommand{\tf}{Julien Sopena, Luciana Arantes, Martin Bertier, Pierre Sens }

\title{\psubject}
\author{Raimana BOUISSOU, Tom GIMENEZ, Léo JECPAS}


\date{}
\begin{document}

\maketitle

%Inclure un fichier tex par "section"
\section{Introduction}
\input{intro}

\section{Algorithme de \nt}
%\section{Fonctionnement de l'algorithme}
%\section{Implémentation}
% 1er Grand paragraphe Algo de Naimi-Tréhel

% Raccourcis...
\newcommand{\al}{"arbre des last" }
\newcommand{\fn}[1]{"file des next" }
\newcommand{\tokr}{<Token Request> }
\newcommand{\tok}{<Token> }


\subsection{Fonctionnement de l'algorithme}
\subsection{Implémentation}
ors


\section{L'extension de \nt permettant la tolérance aux pannes}
%\section{Explication du mécanisme}
%\section{Implémentation}
%\section{Atouts et inconvénients de cette extension}
\subsection{Explication du mécanisme}
\subsection{Implémentation}
\subsection{Atouts et inconvénients de cette extension}



\section{Amélioration de la tolérance aux pannes}
%\section{Les différents mécanismes}
%\section{Implémentation}
\subsection{Les différents mécanismes}
\subsection{Implémentation}


\section{L'algorithme de \nt sur des exemples}
%\section{\nt sans extension}
%\section{\nt avec extension}
\subsection{\nt  sans extension}
\subsection{\nt  avec extension}



\section{Conclusion}
\input{conclusion}

\end{document}
